\documentclass[a4paper,12pt]{article}
\usepackage[utf8]{inputenc}
\usepackage{amsmath}
\usepackage{geometry}
\usepackage{amssymb}
\usepackage{amsthm}
\addtolength{\topmargin}{-.875in}
\usepackage{natbib}
\usepackage{graphicx}

\textheight = 674 pt
\begin{document}

\title{\textbf{LASER: An Investigation}}
\author{By Beshoy Melad Atef - Mina Zaher Shehad \\ Ahmed Tarek Musa - Mina Maged Mounir  \\ Anas Abdallah Youssef(12.B)}
\date{April 2020}
\maketitle
\section{Definition}
A \textbf{LASER} is a device that emits light through a process of optical amplification based on the \textbf{stimulated emission} of electromagnetic radiation.\\ \\
The term “\textbf{LASER}" originated as an acronym for \\ “\textit{\textbf{L}}ight \textit{\textbf{A}}mplification by \textit{\textbf{S}}timulated \textit{\textbf{E}}mission of \textit{\textbf{R}}adiation".
\section{Mechanism (Emission)}
An atom is at first at a stable state called \textbf{ground} state (with energy $E_1$) . \\ \\
When the atom receives a photon of energy, it goes to a higher state which is called an \textbf{excited} level of the atom. \\ \\
\fbox{\begin{minipage}{35em}
\textbf{Life of excitation}: of an atom is a short period of time where an atom loses its excitation and it equals ${10}^{-8}$ seconds.
\end{minipage}} \\ \\
There are two main ways an atom loses its excitation:  
\subsection{Spontaneous emission}
When the excited atom relaxes from the excited state at the end of the lifetime, the atom emits a photon whose energy is equal to the difference between the two states, as in Figure $1$.

\subsection{Stimulated emission}
Stimulated emission is the process by which an incoming photon of a specific frequency can interact with an excited atomic electron (or other excited molecular state), causing it to drop to a lower energy level.
When the excited atom relaxes from the excited state to a lower energy state before the end of its lifetime due to falling of a photon its energy equals the difference between the energies of the two states, the atom is stimulated to emit a photon in phase with the incident photon, as in Figure $1$.
\begin{figure}[h!]
\centering
\includegraphics[scale=1.3]{emission.png}
\caption{Mechanism of Emission}
\label{fig:emission}
\end{figure}\\
The emitted light in both has certain properties which we shall discuss:
\subsection{Properties of the emitted light}
\begin{enumerate}
  \item Two coherent photons having the same frequency propagate in the same direction and the same phase. 
  \item The emitted photons are monochromatic. 
  \item The emitted photons propagate in one direction in the form of parallel rays.  
  \item The intensity of radiation remains constant during propagation for long distances as it doesn't obey the inverse square law ($I\propto \frac{1}{d^2}$). 
\end{enumerate}
We will discuss the properties of the LASER specifically in details.
\section{LASER Properties}
\subsection{Monochromatism}
The light emitted from a LASER radiation is monochromatic (or nearly monochromatic), that is, it is of one wavelength (color).  Monochromatic refers to a single wavelength, or “one spectral line” of light. LASER radiation contains a narrow band of wavelengths (narrow band width). The intensity is concentrated at the wavelength of this spectral line. 
\paragraph{}
Monochromatic light is optical radiation containing only a single optical frequency. The associated electric field strength at a certain point in space, for example, exhibits a purely sinusoidal oscillation, having a constant instantaneous frequency and a zero bandwidth. 
\paragraph{}
Real light sources can of course never be exactly monochromatic, i.e., have a zero optical bandwidth. However, particularly LASER sources are often quasi-monochromatic, i.e., the optical bandwidth is small enough that certain behavior of the light can hardly be distinguished from that of truly monochromatic light.
\begin{figure}[h!]
\centering
\includegraphics[scale=0.5]{prism.jpg}
\caption{light}
\label{fig:emission}
\end{figure}
 
\subsection{Coherence}
\begin{itemize}
     \item \textbf{Spatial coherence} is for light impinging on a surface, the light is coherent if the waves (or photons) at any two points selected at random on the plane maintain a constant phase difference over time. LASER light develops a “speckle pattern”, because coherent waves in the beam interfere to produce bright and dark regions in the area illuminated. 
     \item \textbf{Temporal coherence} is the ability of light to maintain a constant phase at one point in space at two different times, separated by delay $\tau_{\text{}}$. Temporal coherence characterizes how well a wave can interfere with itself at two different times and increases as a source becomes more monochromatic.
     \item \textbf{A coherence time} ($\tau_{\text{cor}}$) and coherence length ($c \cdot \tau_{\text{cor}}$, where $c$ is the speed of light) can be calculated from the spread of wavelengths ($\Delta \lambda $), or frequencies ($\Delta \nu$), in a beam. Expressed in terms of $\Delta \nu$, or “bandwidth”:
     $$\tau_{\text{cor}}=\frac{1}{2\pi\Delta \nu}$$
   \end{itemize}
\begin{figure}[h!]
\centering
\includegraphics[scale=0.6]{Coherent light waves.jpg}
\caption{Coherent light waves}
\label{fig:emission}
\end{figure}
 
\subsection{Intensity}
The intensity of a LASER beam at some location is optical power per unit area, which is transmitted through an imagined surface perpendicular to the propagation direction. The units of the optical intensity (or light intensity) are $W/m^2$ or (more commonly) $W/cm^2$. The intensity is the product of photon energy and photon flux. It is sometimes called optical energy flux. In LASER, the light spreads in small region of space and in a small wavelength range. Hence, LASER light has greater intensity when compared to the ordinary light.
\paragraph{}
LASER intensity doesn’t obey the inverse square law as of intensity radiation remains constant while falling on a surface despite whatever increment in distance between the LASER source and the object. This is mainly because of the photons coherency so rays are intense, concentrated, and propagate for far distances without dispersion.
\subsection{Collimation}
By definition, it is the process through which a radiation magnetic waves’ beam lessens divergence and convergence, the degree to which the beam remains parallel with distance. The beam’s diameter stays constant for long distances. The beam, propagating on parallel rays, avoids scattering and reserves energy during transmission without much loses.
 
\section{LASER Components}
Although there are various types of LASER, but they consist of 3 main components:
\begin{enumerate}
    \item Lasing material or active medium
    \begin{itemize}
     \item It is excited by the external energy resource to produce the population inversion. In the Active medium that stimulated emission of photons takes place, prompting the phenomenon of optical gain, or amplification.
     \item Examples: Crystalline solid such as ruby, Semi-conductors such as silicon crystals, and gas molecules such as Carbon dioxide 
   \end{itemize}
   \begin{figure}[h!]
   \centering
   \includegraphics[scale=0.8]{LASER.png}
   \caption{LASER medium}
   \label{fig:emission}
   \end{figure}
   \item External energy source
   \begin{itemize}
       \item It is used to provide atoms or molecules of the active medium with the required energy, it can be electrical, optical, thermal or chemical energy, to excite the lasing material to produce the population inversion.
   \end{itemize}
   \item Optical resonator
   \begin{itemize}
       \item It essentially gives direction about the stimulated emission process. It is instigated by rapid photons. At long last, a LASER bar will be produced.
   \end{itemize}
\end{enumerate}
 
\section{Theory of the LASER action}
\begin{enumerate}
    \item LASER action depends upon forcing the atoms or molecules of the lasing material to a state of inversion population inversion, in this case, the number of excited atoms is more than the number of ground-state atoms.
    \item The emission of photons develops excited atoms by stimulated emission.
    \item The emitted radiation via simulated emission could be amplified inside the resonant cavity space where successive reflections arise between the surfaces of the mirrors to stimuli other atoms along its bath to produce new photons, therefore, amplified radiation is produced by means of stimulated emission.
\end{enumerate}
From the previous, we conclude that there are different types of LASER:
\begin{itemize}
    \item Solid LASERs such as ruby.
    \item Liquid LASERs such as liquid dye LASER.
    \item Gas LASERs such as (helium - neon) LASER and argon LASER.
\end{itemize}
 
\section{LASER Applications}
\subsection{Transmission and processing of information}
\begin{itemize}
    \item \textbf{LASER scanners}
    \\ The ability to cognizance LASER beams onto very small spots and to switch them on and off billions of times per second makes LASERs critical tools in telecommunications and records processing. In LASER supermarket scanners, a rotating mirror scans a purple beam while clerks pass packages across the beam. Optical sensors discover light contemplated from striped bar codes on programs, decode the symbol, and relay the statistics to a PC so that it can add the price to the invoice. 
    \begin{figure}[h!]
   \centering
   \includegraphics[scale=0.5]{LASER scanner.jpg}
   \caption{LASER scanner}
   \label{fig:emission}
   \end{figure}
   \item \textbf{Fiber-optics}
   \\ Fiber-optic communication systems that transmit signals a number of kilometers also use semiconductor LASER beams. The optical signals are dispatched at infrared wavelengths of $1.3$ to at $1.6$ micrometers, in which glass fibers are maximum transparent.
   \item \textbf{CDs}
   \item \textbf{LASER printers}
\end{itemize}
\subsection{Military fields}
LASER are used in rocket guidance with high accuracy and in smart bombs and LASER radars, also known as LADAR, where LASERs are used to destroy missiles and planes in space after launch directly.
\subsection{Medicine fields}
LASER rays are used in diagnosis and treatment by endoscopy and also used in ophthalmology.
\begin{itemize}
    \item LASERs are to treat cases of long and short sightedness, so patients can dispose with glasses 
    \item LASERS are used in endoscopy using optical fibers in operative surgery and diagnosis .
    \item LASER thermal heat also used in treatment of retinal detachment.
\end{itemize}
\\
References used: \cite{G1} \cite{G3} \cite{G4} \cite{G5} \cite{G6} \cite{G7}
\newpage
\bibliographystyle{plain}
\bibliography{bib.bib}
\end{document}

% “